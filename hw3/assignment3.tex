\documentclass[10pt,drafclsnofoot,onecolumn]{article}
\usepackage{graphicx}
\usepackage{anysize}
\usepackage{color}
\usepackage{balance}
\usepackage{enumitem}
\usepackage{listings}
\usepackage{cite}
\usepackage{hyperref}
\usepackage{url}      
\usepackage[margin=0.75in]{geometry}
\usepackage{amsmath}
\marginsize{.75in}{.75in}{.75in}{.75in}
             
\begin{document}               
\begin{titlepage}
\centering
{\huge\bfseries Project III: The Kernel Crypto API\par}
\vspace{.5cm}
{\scshape Alex Ruef and David Skeen \par}
\vspace{.5cm}   
{\scshape CS 444 Operating Systems II\par}
\vspace{.5cm} 
{\Large\itshape Fall 2017\par}
\par
\par
\begin{abstract}
In this project we are instructed with creating our own RAM disk device driver.
We must also use Linux's crypto API to encrypt and decrypt our block device.
The device driver must be written as a module that can be moved into a running VM.
In doing so we will learn more about I/O devices and how they interact with the Linux kernel.

\end{abstract}
\end{titlepage}

\section{Main Point}
The main point of this project was to get us familiar with I/O block devices.
By writing our own disk device driver we can learn a lot about what goes into a block device in Linux as well as other operating systems.
Cryptography is another important part of I/O devices that we must learn how to use to implement into our project.

\section{Approach}
Our first step was to gather as much information we could about block devices in the Linux kernel.
We used chapter 16 of Linux Device Drivers as the basis of our device driver\cite{linux_device_drivers}.
The next step was to understand and learn how to use the built in Linux crypto.
Then we just had to combine the previous two steps into our own device driver.

\section{Ensurance of Correction}
To ensure correction of code, follow these steps:

\begin{itemize}
\item Clone the code to a fresh repo: git clone git://git.yoctoproject.org/linux-yocto-3.19
\item Checkout the correct version: git checkout v3.19.2
\item Apply the patch file : git apply assign3.patch
\item Source the correct enviroment: source /scratch/files/environment-setup-i586-poky-linux.csh
\item Copy the core file into the linux folder: cp /scratch/files/core-image-lsb-sdk-qemux86.ext4 .
\item Copy the config file into the linux folder: cp /scratch/files/config-3.19.2-yocto-standard .config
\item Go into the menuconfig and set sbd to a Module:
	\begin{itemize}
	\item make menuconfig
	\item Go to device drivers
	\item Enter block drivers
	\item Go to SBD block driver
	\item Press M
	\item Save and exit
	\end{itemize}
\item Make the VM: make -j4 all
\item Start the vm: qemu-system-i386 -nographic -kernel arch/x86/boot/bzImage -drive file=core-image-lsb-sdk-qemux86.ext4 -enable-kvm -usb -localtime --no-reboot --append "root=/dev/hda rw console=ttyS0 debug"
\item Use the root account with no password
\item Go to the base directory: cd ../../
\item Copy the module into the base directory using scp from the drivers/block/ directory: EXAMPLE: scp skeenda@os2.engr.oregonstate.edu:/scratch/fall2017/16/linux-yocto-3.19/drivers/block/sbd.ko .
\item Start up the module: insmod sbd.ko
\item Create the ext2 file directory: mkfs.ext2 /dev/sbd0
\item Create a directory to mount to: mkdir /mnt
\item Mount it: mount /dev/sbd0 /mnt
\item Enter some text into a random file on the mounted dir: echo "Testing 123 Can you find me" > /mnt/test
\item Try to find that text on the device: grep -a "Testing" /dev/sbd0
\item If nothing is returned for the previous step, then there has been a succesful encryption. If there was no encryption that occured, then you would be able to find the text via grepping. You can try any other things that are put into the file but as long as they do not show up in the grep, we have encryption.

\end{itemize}

\section{What did you Learn?}
We learned about I/O devices and how to implement them in the Linux kernel.
While writting the I/O device we learned a lot about how they work and about design decisions that can be made.
Using crypto.h we learned how to encrypt and decrypt blocks of data.
In the process of using crypto.h we also learned how to read and understand code with poor documentation.
Our driver device had to be setup as a module so we had to learn what that was and how to use it in the Linux kernel.

\section{Git Log:}
All commits were to https://github.com/skeenda/CS44-OSU/commit\\
\\
\begin{tabular}{|p{1.5cm}|p{2cm}|p{5cm}|p{7cm}|}
\textbf{Detail} & \textbf{Author} & \textbf{Date} & \textbf{Description}\\\hline
29de258 & Alex Ruef & Sun Nov 12 15:56:19 2017 -0800 & copied over makefile from previous assignments\\\hline
2bb4245 & Alex Ruef & Sun Nov 12 15:56:04 2017 -0800 & fixed up git log section\\\hline
1073b6e & Alex Ruef & Sun Nov 12 15:52:49 2017 -0800 & fixed error in bib\\\hline
99ed2cc & Alex Ruef & Sun Nov 12 15:36:04 2017 -0800 & included git log\\\hline
2a5eebb & Alex Ruef & Sun Nov 12 15:33:12 2017 -0800 & made script for easy git log formatting\\\hline
0f5f4a2 & David Skeen & Sun Nov 12 01:55:54 2017 -0800 & Adding the workin linux. The patch has been generated. Added steps to test to the tex\\\hline
ecc8e10 & David Skeen & Sat Nov 11 18:24:28 2017 -0800 & Fixed the last error to allow the code to compile\\\hline
326319a & Alex Ruef & Sat Nov 11 17:10:52 2017 -0800 & wrote abstract\\\hline
1bd7e66 & David Skeen & Sat Nov 11 17:03:22 2017 -0800 & Fixed more errors\\\hline
84ce5ea & David Skeen & Sat Nov 11 16:59:40 2017 -0800 & Fixed some issues with encrypter.c\\\hline
e66c893 & Alex Ruef & Sat Nov 11 16:57:20 2017 -0800 & finished what we learn\\\hline
ab5b231 & Alex Ruef & Sat Nov 11 16:24:36 2017 -0800 & finished approach\\\hline
6ea5f53 & Alex Ruef & Sat Nov 11 16:07:23 2017 -0800 & added bib file\\\hline
557bd4a & David Skeen & Sat Nov 11 15:49:26 2017 -0800 & Added the decrypter and init and free for cipher\\\hline
7ff83e5 & Alex Ruef & Sat Nov 11 15:07:10 2017 -0800 & started writing on our approach\\\hline
0e7c5a1 & Alex Ruef & Sat Nov 11 14:56:38 2017 -0800 & wrote main point\\\hline
ba5bf90 & Alex Ruef & Sat Nov 11 14:37:12 2017 -0800 & initial commit for writeup\\\hline
8fd041e & David Skeen & Fri Nov 10 22:40:40 2017 -0800 & Added the encyrpt block section to the encrypter.c. Added the where I found the block encrypter info in explanation.txt. Took about an hour and half to do both.\\\hline
bea8fb8 & David Skeen & Thu Nov 9 18:24:13 2017 -0800 & Added the Linux crypto header, the key, the key len and the cypher. Need to research where to start the cypher.\\\hline
3e34787 & David Skeen & Thu Nov 9 18:06:27 2017 -0800 & Added a folder for HW3\\\hline
406f33b & David Skeen & Thu Nov 9 18:05:38 2017 -0800 & Added the basic file for the device driver. Need to add encryption to it\\\hline
\end{tabular}

\bibliographystyle{ieeetr}
\bibliography{bib.bib}


\end{document}
