\documentclass[10pt,drafclsnofoot,onecolumn]{article}
\usepackage{graphicx}
\usepackage{anysize}
\usepackage{color}
\usepackage{balance}
\usepackage{enumitem}
\usepackage{listings}
\usepackage{cite}
\usepackage{hyperref}
\usepackage{url}      
\usepackage[margin=0.75in]{geometry}
\usepackage{amsmath}
\marginsize{.75in}{.75in}{.75in}{.75in}
             
\begin{document}               
\begin{titlepage}
\centering
{\huge\bfseries Project III: Getting Acquainted\par}
\vspace{.5cm}
{\scshape Alex Ruef and David Skeen \par}
\vspace{.5cm}   
{\scshape CS 444 Operating Systems II\par}
\vspace{.5cm} 
{\Large\itshape Fall 2017\par}
\par
\par
\begin{abstract}
In this project we are instructed with creating our own RAM disk device driver.
We must also use Linux's crypto API to encrypt and decrypt our block device.
The device driver must be written as a module that can be moved into a running VM.
In doing so we will learn more about I/O devices and how they interact with the Linux kernel.

\end{abstract}
\end{titlepage}

\section{Main Point}
The main point of this project was to get us familiar with I/O block devices.
By writing our own disk device driver we can learn a lot about what goes into a block device in Linux as well as other operating systems.
Cryptography is another important part of I/O devices that we must learn how to use to implement into our project.

\section{Approach}
Our first step was to gather as much information we could about block devices in the Linux kernel.
We used chapter 16 of Linux Device Drivers as the basis of our device driver.
The next step was to understand and learn how to use the built in Linux crypto.
Then we just had to combine the previous two steps into our own device driver.

\section{Ensurance of Correction}
To ensure correction of code, follow these steps:

\begin{itemize}
\item Clone the code to a fresh repo: git clone git://git.yoctoproject.org/linux-yocto-3.19
\item Checkout the correct version: git checkout v3.19.2
\item Apply the patch file : git apply assign3.patch
\item Source the correct enviroment: source /scratch/files/environment-setup-i586-poky-linux.csh
\item Copy the core file into the linux folder: cp /scratch/files/core-image-lsb-sdk-qemux86.ext4 .
\item Copy the config file into the linux folder: cp /scratch/files/config-3.19.2-yocto-standard .config
\item Go into the menuconfig and set sbd to a Module:
	\begin{itemize}
	\item make menuconfig
	\item Go to device drivers
	\item Enter block drivers
	\item Go to SBD block driver
	\item Press M
	\item Save and exit
	\end{itemize}
\item Make the VM: make -j4 all
\item Start the vm: qemu-system-i386 -nographic -kernel arch/x86/boot/bzImage -drive file=core-image-lsb-sdk-qemux86.ext4 -enable-kvm -usb -localtime --no-reboot --append "root=/dev/hda rw console=ttyS0 debug"
\item Use the root account with no password
\item Go to the base directory: cd ../../
\item Copy the module into the base directory using scp from the drivers/block/ directory: EXAMPLE: scp skeenda@os2.engr.oregonstate.edu:/scratch/fall2017/16/linux-yocto-3.19/drivers/block/sbd.ko .
\item Start up the module: insmod sbd.ko
\item Create the ext2 file directory: mkfs.ext2 /dev/sbd0
\item Create a directory to mount to: mkdir /mnt
\item Mount it: mount /dev/sbd0 /mnt
\item Enter some text into a random file on the mounted dir: echo "Testing 123 Can you find me" > /mnt/test
\item Try to find that text on the device: grep -a "Testing" /dev/sbd0
\item If nothing is returned for the previous step, then there has been a succesful encryption. If there was no encryption that occured, then you would be able to find the text via grepping. You can try any other things that are put into the file but as long as they do not show up in the grep, we have encryption.

\end{itemize}

\section{What did you Learn?}
We learned about I/O devices and how to implement them in the Linux kernel.
While writting the I/O device we learned a lot about how they work and about design decisions that can be made.
Using crypto.h we learned how to encrypt and decrypt blocks of data.
In the process of using crypto.h we also learned how to read and understand code with poor documentation.
Our driver device had to be setup as a module so we had to learn what that was and how to use it in the Linux kernel.

\section{Work Log}
\begin{tabular}{|p{5cm}|p{5cm}|p{5cm}}
\textbf{Approximate Start Time} & \textbf{Approximate Duration} & \textbf{Activity} \\
\hline
9:15 Wed, Oct 4 & 7 minutes & Made a script for execution \\\hline
9:22 Wed, Oct 4 & 1 hour & Copied all necessary files and cloned the git directory. Edited the script for correct run. \\\hline
10:30 Wed, Oct 4 & 15 minutes & Started the VM and figured out how to connect to it \\\hline
10:45 Wed, Oct 4 & 30 minutes & Searched for the correct file in the bin to run the new VM with stuff generated from make -j4 all. Made the changes in the script and then ran the vm to make sure it would work \\
10:15 Thu, Oct 5 & 1 hr 45 minutes & Read the little book of Semaphores solution. Began writing with pthreads, semaphores, RDRAND, and the Mersenne Twister \\\hline
12:15 Fri, Oct 6 & 1 hr & Got the random number generator in the correct range \\\hline
13:00 Sat, Oct 7 & 2 hr & Began writup in tex \\\hline
16:00 Sat, Oct 7 & 2 hr & RDRAND and Semaphores working correctly \\\hline
18:00 Sat, Oct 7 & 15 min & Made sure everything is running correctly \\\hline
12:00 Sun, Oct 8 & 2 hr & Added most of the writeup, still need Logs and code section \\\hline
20:00 Sun, Oct 8 & 2 hr & Worked on the log portion and the code portion \\\hline
\end{tabular}
\\
\\
\section{Git Log:}

All commits were to https://github.com/skeenda/CS44-OSU/commit\\
Detail has been reduced to last 7 characters of commit\\
\begin{tabular}{|p{5cm}|p{5cm}|p{5cm}}\textbf{Detail} & \textbf{Author} & \textbf{Description}\\
\hline
c7b1a03 & David Skeen & Added a script to run the command given and then copied the enviroment setup to source\\\hline
8ba3373 & David Skeen &  Cloned the directory and then checked out v3.19.2\\\hline
df9f3e0 & David Skeen & Copied the config to my directory\\\hline
f8b73d4 & David Skeen &  Edited the run\_qemu for the correct port and changed back from something that I thought it was supposed to be to the bzImage. Added bzImage to make it run the base.\\\hline
eb1acb2 & Alexander Ruef & initial concurrency exercise commit\\\hline
2851d15 & Alexander Ruef & random number generation now in correct range\\\hline
2e7eda4 & Alex Ruef & Added Queue functionality\\\hline
9cef3b2 & Alex Ruef & working on queue size limit\\\hline
5216009 & Alex Ruef & Semaphores correctly in place\\\hline
8d50405 & Alex Ruef & Added support for RDRAND generation\\\hline
8805969 & Alex Ruef & final concurrency commit\\\hline
9b4906e & Alex Ruef & initial commit for sources\\\hline
c17fee8 & David Skeen & Added files to enable tex and bib to work\\\hline
\end{tabular}

\section{Appendix 1: Concurrency Answer}
\subsection{concurrenc.c}
\begin{lstlisting}

 \end{lstlisting}

\bibliographystyle{ieeetr}
\bibliography{bib.bib}


\end{document}
